\phantomsection\addcontentsline{toc}{section}{BT in Cybersecurity}%
\section*{Bachelor of Technology: Cybersecurity}

\begin{reqgroup}{General Information}
\begin{itemize}\small

	\item The Bachelor of Technology (BT) is a baccalaureate program for individuals with approved associate degrees. You may transfer your approved degree intact up to 89 credit hours and earn an accredited 4-year degree in about two years of full-time work.

	\item Your BT degree is individually tailored based on courses diverse courses transferred in, your goals, course availability, academic viability, etc. Consequently, the BT degree is flexible to your situation and focused on you.

	\item In addition to transferred credit, you must complete: all outstanding core courses, the cognate, and any elective courses to total at least 120 credit hours (including associate degree credit).

	\item To apply for the BT, contact CSU Admissions at:
		\href{https://apply.charlestonsouthern.edu/}{apply.charlestonsouthern.edu} (Undergraduate Admissions), or at (843) 863-7050 or (800) 947-7474.\\
		Have your official transcripts sent to the Office of Admissions (address at the bottom of the page below). Indicate BT on the application.

	\item For more information, ask at: \href{https://www.charlestonsouthern.edu/contact-us/}{charlestonsouthern.edu/contact-us/} .


	\item When the evaluation is done, the Registrar will assign you to an advisor. Contact your advisor to plan your BT degree.
\end{itemize}
\end{reqgroup}

\begin{reqgroup}{Liberal Arts Core Requirements (48--50 hours)}
\begin{checklist}
\begin{minipage}[t]{0.5\linewidth}
	\item (3) ENGL 111 or 180 Composition and Rhetoric 1
	\item (3) ENGL 112 Composition and Rhetoric 2
	\item (3) ENGL 202, 203, or 204 Literature
	\item (3) COMM 110 Public Speaking
	\item (3) Any 3-hour foreign language course
	\item (3--4) Any CSCI course
	\item (3) CHST 111  Survey of the Old Testament
	\item (3) CHST 112  Survey of the New Testament
	\item (3) HIST 111, 112, or 113 Western Civ.
\end{minipage}
\begin{minipage}[t]{0.5\linewidth}
	\item (3) Another History, POLI 201, or CHST 140
	\item (3) ART 202 Art Appr, or MUSI 171 or 371, or THEA 218 or 311
	\item (3) BUSI 203, CRIM 210, ECON 211, ECON 212, HEAL 201, POLI 101, PSYC 110, or SOCI 101, 203, or 205
	\item (3--4) \textbf{MATH 110 College Algebra} (or higher)
	\item (4) Lab Science
	\item (4) Lab Science\\Limited one lab science course per category of Biology, Chemistry,
Geology, or Physics.
\end{minipage}
\end{checklist}
\end{reqgroup}

%\vspace{1em}

\begin{reqgroup}{Cybersecurity Cognate (19 hours)}%
\begin{checklist}%
\begin{minipage}[t]{0.5\linewidth}%
	\item (4) CSCI 235 Procedural Programming
	\item (4) CSCI 301 Survey of Scripting Languages\\
		or CSCI 352 Cyber Defense
	\item (4) CSCI 332 Applied Networking
\end{minipage}%
\begin{minipage}[t]{0.5\linewidth}
	\item (3) CSCI 405 Principles of Cybersecurity
	\item (4) CSCI 433 Network Security
	\item (0) CSCI 496 Senior Portfolio Review
\end{minipage}
\end{checklist}
\end{reqgroup}

\begin{reqgroup}{General Electives}
All non-cognate courses used for the required 120-hour graduation minimum, prerequisites, or other requirements.
\end{reqgroup}

Notes:%
\begin{enumerate}\footnotesize
	\item No more than 89 hours from sources other than baccalaureate degree-granting institutions may be used toward the BT degree. However, accepted, non-associate degree credit from baccalaureate institutions might be used.
	\item CSCI 496 Senior Portfolio is required for graduation. See CSCI 496 in the catalog for details.
	\item If prerequisites are needed for the cognate requirements, summer courses may be required for four-years completion.
	\item Residency requirements: 36 of the last 46 hours must be earned while attending CSU. Of these, \textbf{at least 15} must be 300--400 level courses.
	\item Students may request one official degree audit through their \href{https://portal.csuniv.edu/}{MyCSU account}. Recommending during the next to last semester.
	\item BT candidates who miss two consecutive major semesters must revise the degree plan for any new requirements.
	\item Students are responsible for completing an Application for Graduation. See Registrar.
	\item Full-time day students must meet chapel requirements. See the Student Development section of the catalog.
	\item In cases of conflict between this checklist and the catalog, the catalog takes precedence.
\end{enumerate}